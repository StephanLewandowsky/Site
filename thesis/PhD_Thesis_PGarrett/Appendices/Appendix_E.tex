\chapter{Supplementary Materials S5: \\ The cross cultural cost of errors}
\label{Appendix:E_CrossWheelTask}
\lhead{Supplementary S5. \emph{Cross Cultural Errors}}

\setcounter{equation}{0}
\setcounter{figure}{0}
\setcounter{table}{0}
\setcounter{section}{0}
\renewcommand\thefigure{S5\thesection.\arabic{figure}}
\renewcommand\thetable{S5\thesection.\arabic{table}}

\noindent The materials in this supplementary chapter are relevant to Chapter 7 of the submitted thesis.

\newpage

\section{Analysis of the calibration staircase procedure}
\label{Appendix:Staircase}
%\lhead{Supplementary \ref{Appendix:E_CrossWheelTask}. \emph{Staircase procedure}}

The following provides a full analysis of our successful staircase procedure. As the reader will soon learn, our staircase procedure successfully manipulated stimulus accuracy for all participants in all numeric types. Although one participants finished at a relatively high critical contrast level in the non-symbolic dot numeric type, the MDS results of this participant did not appear to differ from the other subjects. We summarised by saying this procedure was highly effective and successfully produced the necessary confusions

Figure \ref{fig:Staircase_Cross} (top) depicts the calibration block for participant S1, responding to Arabic numerals. This staircase procedure was typical of most participants. The mean contrast level of the final 30 assessment trials (highlighted yellow) determined the critical contrast value --- the value from which stimulus-signal levels were determined in the experimental session. Violin plots (bottom) depict the mean and standard-error of contrast values for assessment trials, for each participant and language-type. Critical contrast levels were stable across language-type and participants, except for participant S2 in the non-symbolic dots condition. This participant's staircase procedure resulted in the highest (easiest) critical contrast level, indicating their performance was particularly poor during the calibration block. 

% Staircase Figure
\begin{figure}[tbh]
\centering \includegraphics[width=\linewidth]{Figures/Appendix/AppE/StaircaseAndViolin.jpg}
\caption{Plot of participant S1's Arabic numeral staircase procedure (top) and violin plots of individual participant's staircase assessment trials (bottom). Assessment trials (highlighted yellow for participant S1) determined the critical contrast value for the main experiment. For all language-types, participants displayed relatively stable contrast levels during the assessment window. Black lines on each violin plot represent the critical contrast value (mean RGB value over the assessment window). Colored ticks on the y-axis are the mean critical contrast values in each language-type.}
\label{fig:Staircase_Cross}
\end{figure}

Colored ticks on the violin-plot (Figure \ref{fig:Staircase_Cross}, y-axis) show, on average, critical contrast levels were approximately equal for Chinese numerals (RGB $\mu$ = 137.4, $\sigma$ = 1.49), Arabic numerals (RGB $\mu$ = 137.6, $\sigma$ = 1.15) and non-symbolic dots (RGB $\mu$ = 137.8, $\sigma$ = 3.89); and highest for Thai numerals (RGB $\mu$ = 140.3, $\sigma$ = 1.75). A higher critical contrast level for Thai numerals suggests unfamiliar numeric items were more difficult to identify than familiar numeric items (Chinese, Arabic and Dots). A one-way repeated-measures ANOVA displayed a significant main effect of critical contrast across language-types (\Fval{3}{30}{4.22}{$<$ 0.05}). Post-hoc paired-sample $t$-tests displayed significant differences in critical contrast levels between Arabic and Thai numerals (\tval{10}{-4.217}{$<$ .05}) and Chinese and Thai numerals (\tval{10}{-5.2}{$<$ 0.01}); other comparisons did not reach statistical significance (summarized in Table \ref{tab:ttest_CalibrationMean_Cross}). 


\section{Experimental accuracy by participant, contrast level and numeric type}
\label{Appendix:Accuracy}
%\lhead{Supplementary \ref{Appendix:E_CrossWheelTask}. \emph{Experimental accuracy}}
During experimental trials, stimuli were presented at five signal-levels: one step below the critical contrast value (level 1: hardest) and three steps above (levels 3, 4 and 5: easiest). Across language-types, mean accuracy increased linearly with the visibility of the contrast levels (Figure \ref{fig:AccContrast_Cross}.a). Accuracy was lowest at level 1 ($\mu$ = .48, $\sigma$ = .17) and highest at level 5 ($\mu$ = .79, $\sigma$ = .11). Across experimental trials, mean accuracy was highest for non-symbolic dots ($\mu$ = .72, $\sigma$ = .14), then Chinese numerals ($\mu$ = .66, $\sigma$ = .12), Arabic numerals ($\mu$ = .62, $\sigma$ = .14), and finally, Thai numerals ($\mu$ = .59, $\sigma$ = .11).

\begin{figure}[tbh]
\centering \includegraphics[width = \linewidth]{Figures/Appendix/AppE/AccuracyByContrast.jpg}
\caption{a) Mean accuracy across five signal contrast-levels, and four language-types. b) Mean accuracy across each experimental block. c) Mean accuracy for each participant by critical contrast level. d) Mean accuracy matched by contrast-level, across language-types. Error bars represent the standard-error of the mean.}
\label{fig:AccContrast_Cross}
\end{figure}

A repeated measures ANOVA displayed a main effect of contrast level on accuracy (\Fval{4}{40}{227.262}{$<$ 0.001}, $\eta^2$ = 0.958), but no main effect of language-type on accuracy (\Fval{3}{30}{2.042}{= 0.13}, $\eta^2$ = 0.17). There was a significant interaction of language-type and contrast level on accuracy (\Fval{12}{120}{1.872}{$<$ 0.05}, $\eta^2$ = 0.158). Post-hoc paired $t$-tests displayed significant differences between all combinations of contrast level ($p$ $<$ .001), and no significant differences between comparisons of language-type (simple effects are reported in supplementary material \ref{Appendix:ttests}, Tables \ref{tab:LevelCompare_Cross} and \ref{tab:LangLvlCompare_Cross}. These results indicate our chosen signal levels appropriately influenced response accuracy, and that familiarity had no effect on response-accuracy. We will return to this shortly. 

Figure \ref{fig:AccContrast_Cross}.b. depicts mean accuracy across experimental blocks and language-type. Mean accuracy was comparable between language-types, and increased marginally with block number, being lowest at block 1 ($\mu$ = .59, $\sigma$ = .15) and highest at block 7 ($\mu$ = .67, $\sigma$ = .15), before plateauing to block 13 ($\mu$ = .66, $\sigma$ = .15). A repeated-measures ANOVA displayed a significant main effect of block number on accuracy (\Fval{12}{120}{7.207}{$<$ 0.001}, $\eta^2$ = 0.419), and did not display a significant interaction of block number and language-type on accuracy (\Fval{36}{360}{1.033}{$=$ 0.42}, $\eta^2$ = 0.094). Post-hoc paired-sample $t$-tests displayed significant differences in accuracy between blocks 1 and 9--11 ($p$ $<$ 0.05), and blocks 2 and 10 ($p$ $<$ 0.05; simple effects reported in supplementary material D Table \ref{tab:BlockAcc_Cross}). These results suggest a small practice effect improved accuracy from early to late blocks.

Figure \ref{fig:AccContrast_Cross}.c. presents mean experimental accuracy across critical contrast levels, separated by participant and language-type. A linear regression found a significant positive relationship between critical contrast and mean accuracy ($r^2$ = .525), suggesting a dependency between contrast and accuracy. To disentangle the effect of language-type and contrast on accuracy, we assessed accuracy matched across RGB values from each participant's five signal-contrast levels (see \ref{fig:AccContrast_Cross}.d).

Figure \ref{fig:AccContrast_Cross}.d. presents mean accuracy matched across participant's five contrast-levels, separated by language-type. For example, if for Arabic numerals, participant S1 responded to RGB contrast values 130--134 and participant S2 responded to RGB contrast values 134--137, their accuracy at contrast value 134 would be averaged and depicted in Figure \ref{fig:AccContrast_Cross}.d. 

Figure \ref{fig:AccContrast_Cross}.d. displays a positive relationship between contrast and matched accuracy. Matching accuracy for contrast levels when all language-types were presented (\emph{i.e.,} excluding contrast values $<$ 135 and $>$ 140), accuracy was highest for non-symbolic dots ($\mu$ = .74, $\sigma$ = .06), then Chinese numerals ($\mu$ = .71, $\sigma$ = .09), then Arabic numerals ($\mu$ = .66, $\sigma$ = .11) and lowest for Thai numerals ($\mu$ = .61, $\sigma$ = .09). 

We completed a two-way between-subjects ANOVA to assess the effect of language-type and contrast-level on matched accuracy (Figure \ref{fig:AccContrast_Cross}.d). We found a main effect of language-type (\Fval{3}{141}{7.458}{$<$ .001}, $\eta^2$ = 0.08), and a main effect of contrast-level (\Fval{5}{141}{24.252}{$<$ .001}, $\eta^2$ = 0.42) on accuracy.  There was no interaction effect between contrast level and language-type on accuracy (\Fval{15}{141}{0.512}{= .93}, $\eta^2$ = 0.03). Post-hoc $t$-tests displayed significant differences between all contrast vales ($p$ $<$ .05), except comparisons made between high RGB values: 137 and 138, 138 and 139, 138 and 140 and 139 and 140 (pair-wise tests are reported in the supplementary Table \ref{tab:ttest_MatchedContrastAcc_Cross}). This suggests a plateau in accuracy occurring at high RGB contrast values. Post-hoc $t$-tests displayed significant differences in matched accuracy between Arabic and Thai numerals (\tval{3}{3.969}{$<$ 0.001}) and non-symbolic dots and Thai numerals (\tval{3}{3.988}{$<$ 0.001}); other comparisons did not reach statistical significance (summarised in supplementary material D Table \ref{tab:LangMatchedAcc_Cross}). When accuracy was matched by contrast level, the familiar Arabic and Dot numeric-sets were more accurately reported than the Thai numeric-set. Familiar Chinese numerals tended to be more accurately reported than Thai numerals, however, this was not borne out at the statistical level. 


\pagebreak
%Prefix a "S" to all equations, figures, tables and reset the counter
% \setcounter{equation}{0}
% \setcounter{figure}{0} 
% \setcounter{table}{0}
% \setcounter{page}{1}
% \makeatletter
% \renewcommand{\appendixname}{Supplementary Material}
% \renewcommand{\theappendix}{S\arabic{section}}
% \makeatother
% \newpage
% \appendix


\section{Simple effects: \textit{t}-tests}
\label{Appendix:ttests}
%\lhead{Supplementary \ref{Appendix:E_CrossWheelTask}. \emph{Simple Effects}} 

\begin{table}[ht]
	\centering
	\caption{Post Hoc Comparisons - CalibrationMeans}
	\begin{tabular}{lrrrrrr}
		\hline
		 &  & Mean Difference & SE & t & Cohen's d & p$_{bonf}$  \\
		\hline
		ENG & DOT & -0.215 & 1.203 & -0.179 & -0.054 & 1.000  \\
		  & CHN & 0.227 & 0.641 & 0.354 & 0.107 & 1.000  \\
		 & THI & -2.691 & 0.638 & -4.217 & -1.272 & 0.011  \\
		DOT & CHN & 0.442 & 0.974 & 0.454 & 0.137 & 1.000  \\
		  & THI & -2.476 & 1.314 & -1.884 & -0.568 & 0.534  \\
		CHN & THI & -2.918 & 0.561 & -5.200 & -1.568 & 0.002  \\
		\hline
	\end{tabular} 
	\label{tab:ttest_CalibrationMean_Cross}
\end{table}

\begin{table}[ht]
	\centering
	\caption{Post Hoc Comparisons - Contrast Levels}
	\begin{tabular}{lrrrrrr}
		\hline
		 &  & Mean Difference & SE & t & Cohen's d & p$_{bonf}$  \\
		\hline
		L1 & L2 & -0.087 & 0.011 & -7.985 & -2.408 & $<$ .001  \\
		  & L3 & -0.182 & 0.010 & -18.390 & -5.545 & $<$ .001  \\
		 & L4 & -0.253 & 0.014 & -18.701 & -5.639 & $<$ .001  \\
		 & L5 & -0.313 & 0.017 & -18.382 & -5.542 & $<$ .001  \\
		L2 & L3 & -0.095 & 0.010 & -9.846 & -2.969 & $<$ .001  \\
		  & L4 & -0.166 & 0.012 & -14.352 & -4.327 & $<$ .001  \\
		 & L5 & -0.226 & 0.015 & -14.633 & -4.412 & $<$ .001  \\
		L3 & L4 & -0.072 & 0.008 & -9.340 & -2.816 & $<$ .001  \\
		  & L5 & -0.132 & 0.011 & -11.762 & -3.546 & $<$ .001  \\
		L4 & L5 & -0.060 & 0.007 & -8.183 & -2.467 & $<$ .001  \\
		\hline
	\end{tabular} 
	\label{tab:LevelCompare_Cross}
\end{table}

\begin{table}[ht]
	\centering
	\caption{Post Hoc Comparisons - Numeric type}
	\begin{tabular}{lrrrrrr}
		\hline
		 &  & Mean Difference & SE & t & Cohen's d & p$_{bonf}$  \\
		\hline
		ARABIC & CHINESE & -0.034 & 0.068 & -0.494 & -0.149 & 1.000  \\
		  & THAI & -0.133 & 0.064 & -2.089 & -0.630 & 0.380  \\
		 & DOTS & -0.076 & 0.063 & -1.202 & -0.362 & 1.000  \\
		CHINESE & THAI & -0.100 & 0.048 & -2.063 & -0.622 & 0.397  \\
		  & DOTS & -0.042 & 0.034 & -1.241 & -0.374 & 1.000  \\
		THAI & DOTS & 0.058 & 0.057 & 1.006 & 0.303 & 1.000  \\
		\hline
	\end{tabular} 
	\label{tab:LangLvlCompare_Cross}
\end{table}

\newpage
\begin{longtable}{lrrrrrr}
\caption{Post-hoc comparisons of accuracy by block number.}
\label{tab:BlockAcc_Cross}\\
	\hline
	 &  & Mean Difference & SE & t & Cohen's $d$ & $p_{bonf}$ \\
	\hline
	\endfirsthead
	
	\multicolumn{7}{c}%
    {{\bfseries Table \thetable\ continued from previous page}} \\
    \hline
    &  & Mean Difference & SE & t & Cohen's $d$ & $p_{bonf}$ \\
    \hline
    \endhead
	
	Block 1 & Block 2 & -0.014 & 0.016 & -0.878 & -0.265 & 1.000  \\
	& Block 3 & -0.034 & 0.016 & -2.195 & -0.662 & 1.000  \\
	& Block 4 & -0.049 & 0.011 & -4.452 & -1.342 & 0.096  \\
	& Block 5 & -0.043 & 0.016 & -2.702 & -0.815 & 1.000  \\
	& Block 6 & -0.070 & 0.015 & -4.839 & -1.459 & 0.053  \\
	& Block 7 & -0.074 & 0.016 & -4.711 & -1.421 & 0.065  \\
	& Block 8 & -0.072 & 0.017 & -4.142 & -1.249 & 0.156  \\
	& Block 9 & -0.068 & 0.014 & -5.013 & -1.512 & 0.041  \\
		 & Block 10 & -0.072 & 0.014 & -5.141 & -1.550 & 0.034  \\
		 & Block 11 & -0.063 & 0.013 & -4.885 & -1.473 & 0.050  \\
		 & Block 12 & -0.067 & 0.016 & -4.156 & -1.253 & 0.153  \\
		 & Block 13 & -0.070 & 0.015 & -4.624 & -1.394 & 0.074  \\
		Block 2 & Block 3 & -0.021 & 0.011 & -1.817 & -0.548 & 1.000  \\
		  & Block 4 & -0.036 & 0.012 & -2.900 & -0.874 & 1.000  \\
		 & Block 5 & -0.030 & 0.015 & -1.959 & -0.591 & 1.000  \\
		 & Block 6 & -0.057 & 0.016 & -3.603 & -1.086 & 0.376  \\
		 & Block 7 & -0.061 & 0.016 & -3.851 & -1.161 & 0.250  \\
		 & Block 8 & -0.059 & 0.016 & -3.612 & -1.089 & 0.371  \\
		 & Block 9 & -0.055 & 0.015 & -3.728 & -1.124 & 0.306  \\
		 & Block 10 & -0.059 & 0.011 & -5.214 & -1.572 & 0.031  \\
		 & Block 11 & -0.050 & 0.013 & -3.732 & -1.125 & 0.304  \\
		 & Block 12 & -0.053 & 0.017 & -3.129 & -0.943 & 0.835  \\
		 & Block 13 & -0.056 & 0.015 & -3.842 & -1.158 & 0.254  \\
		Block 3 & Block 4 & -0.015 & 0.014 & -1.062 & -0.320 & 1.000  \\
		  & Block 5 & -0.009 & 0.020 & -0.435 & -0.131 & 1.000  \\
		 & Block 6 & -0.036 & 0.019 & -1.943 & -0.586 & 1.000  \\
		 & Block 7 & -0.040 & 0.019 & -2.127 & -0.641 & 1.000  \\
		 & Block 8 & -0.038 & 0.019 & -2.025 & -0.611 & 1.000  \\
		 & Block 9 & -0.034 & 0.017 & -1.968 & -0.593 & 1.000  \\
		 & Block 10 & -0.038 & 0.017 & -2.217 & -0.668 & 1.000  \\
		 & Block 11 & -0.029 & 0.018 & -1.630 & -0.491 & 1.000  \\
		 & Block 12 & -0.033 & 0.021 & -1.563 & -0.471 & 1.000  \\
		 & Block 13 & -0.035 & 0.019 & -1.870 & -0.564 & 1.000  \\
		Block 4 & Block 5 & 0.006 & 0.009 & 0.688 & 0.207 & 1.000  \\
		  & Block 6 & -0.021 & 0.011 & -1.982 & -0.598 & 1.000  \\
		 & Block 7 & -0.025 & 0.010 & -2.479 & -0.747 & 1.000  \\
		 & Block 8 & -0.023 & 0.010 & -2.347 & -0.708 & 1.000  \\
		 & Block 9 & -0.019 & 0.009 & -2.145 & -0.647 & 1.000  \\
		 & Block 10 & -0.023 & 0.009 & -2.451 & -0.739 & 1.000  \\
		 & Block 11 & -0.014 & 0.010 & -1.399 & -0.422 & 1.000  \\
		 & Block 12 & -0.018 & 0.013 & -1.383 & -0.417 & 1.000  \\
		 & Block 13 & -0.020 & 0.012 & -1.765 & -0.532 & 1.000  \\
		Block 5 & Block 6 & -0.027 & 0.010 & -2.824 & -0.851 & 1.000  \\
		  & Block 7 & -0.031 & 0.011 & -2.905 & -0.876 & 1.000  \\
		 & Block 8 & -0.029 & 0.011 & -2.567 & -0.774 & 1.000  \\
		 & Block 9 & -0.025 & 0.008 & -3.105 & -0.936 & 0.870  \\
		 & Block 10 & -0.029 & 0.009 & -3.371 & -1.016 & 0.555  \\
		 & Block 11 & -0.020 & 0.011 & -1.901 & -0.573 & 1.000  \\
		 & Block 12 & -0.024 & 0.012 & -2.032 & -0.613 & 1.000  \\
		 & Block 13 & -0.027 & 0.011 & -2.330 & -0.702 & 1.000  \\
		Block 6 & Block 7 & -0.004 & 0.010 & -0.391 & -0.118 & 1.000  \\
		  & Block 8 & -0.002 & 0.013 & -0.141 & -0.042 & 1.000  \\
		 & Block 9 & 0.002 & 0.004 & 0.623 & 0.188 & 1.000  \\
		 & Block 10 & -0.002 & 0.009 & -0.232 & -0.070 & 1.000  \\
		 & Block 11 & 0.007 & 0.011 & 0.673 & 0.203 & 1.000  \\
		 & Block 12 & 0.004 & 0.012 & 0.305 & 0.092 & 1.000  \\
		 & Block 13 & 7.576e-4 & 0.013 & 0.058 & 0.018 & 1.000  \\
		Block 7 & Block 8 & 0.002 & 0.005 & 0.370 & 0.112 & 1.000  \\
		  & Block 9 & 0.006 & 0.010 & 0.597 & 0.180 & 1.000  \\
		 & Block 10 & 0.002 & 0.009 & 0.208 & 0.063 & 1.000  \\
		 & Block 11 & 0.011 & 0.008 & 1.282 & 0.387 & 1.000  \\
		 & Block 12 & 0.007 & 0.007 & 1.058 & 0.319 & 1.000  \\
		 & Block 13 & 0.005 & 0.010 & 0.457 & 0.138 & 1.000  \\
		Block 8 & Block 9 & 0.004 & 0.012 & 0.329 & 0.099 & 1.000  \\
		  & Block 10 & -2.525e-4 & 0.010 & -0.025 & -0.008 & 1.000  \\
		 & Block 11 & 0.009 & 0.010 & 0.888 & 0.268 & 1.000  \\
		 & Block 12 & 0.005 & 0.009 & 0.623 & 0.188 & 1.000  \\
		 & Block 13 & 0.003 & 0.010 & 0.247 & 0.075 & 1.000  \\
		Block 9 & Block 10 & -0.004 & 0.009 & -0.500 & -0.151 & 1.000  \\
		  & Block 11 & 0.005 & 0.011 & 0.433 & 0.131 & 1.000  \\
		 & Block 12 & 0.001 & 0.012 & 0.106 & 0.032 & 1.000  \\
		 & Block 13 & -0.002 & 0.012 & -0.125 & -0.038 & 1.000  \\
		Block 10 & Block 11 & 0.009 & 0.005 & 1.695 & 0.511 & 1.000  \\
		  & Block 12 & 0.006 & 0.011 & 0.516 & 0.156 & 1.000  \\
		 & Block 13 & 0.003 & 0.009 & 0.311 & 0.094 & 1.000  \\
		Block 11 & Block 12 & -0.004 & 0.011 & -0.328 & -0.099 & 1.000  \\
		  & Block 13 & -0.006 & 0.008 & -0.759 & -0.229 & 1.000  \\
		Block 12 & Block 13 & -0.003 & 0.011 & -0.246 & -0.074 & 1.000  \\

	\hline\hline
\end{longtable} 

\newpage

\begin{table}[ht]
	\centering
	\caption{Post Hoc Comparisons - Contrast Steps (RGB)}
	\begin{tabular}{lrrrrrr}
		\hline
		 &  & Mean Difference & SE & t & Cohen's d & p$_{bonf}$  \\
		\hline
		135 & 136 & -0.095 & 0.030 & -3.146 & -0.780 & 0.030  \\
		  & 137 & -0.184 & 0.029 & -6.277 & -1.638 & $<$ .001  \\
		 & 138 & -0.248 & 0.030 & -8.249 & -2.221 & $<$ .001  \\
		 & 139 & -0.302 & 0.039 & -7.757 & -2.787 & $<$ .001  \\
		 & 140 & -0.315 & 0.043 & -7.398 & -3.005 & $<$ .001  \\
		136 & 137 & -0.089 & 0.026 & -3.482 & -0.742 & 0.010  \\
		  & 138 & -0.154 & 0.027 & -5.784 & -1.273 & $<$ .001  \\
		 & 139 & -0.207 & 0.036 & -5.716 & -1.735 & $<$ .001  \\
		 & 140 & -0.221 & 0.040 & -5.490 & -1.867 & $<$ .001  \\
		137 & 138 & -0.064 & 0.026 & -2.499 & -0.573 & 0.204  \\
		  & 139 & -0.118 & 0.036 & -3.304 & -1.077 & 0.018  \\
		 & 140 & -0.131 & 0.040 & -3.311 & -1.231 & 0.018  \\
		138 & 139 & -0.053 & 0.036 & -1.471 & -0.493 & 1.000  \\
		  & 140 & -0.067 & 0.040 & -1.664 & -0.637 & 1.000  \\
		139 & 140 & -0.014 & 0.047 & -0.288 & -0.139 & 1.000  \\
		\hline
	\end{tabular} 
	\label{tab:ttest_MatchedContrastAcc_Cross}
\end{table}'

\begin{table}[ht]
	\centering
	\caption{Post Hoc Comparisons - Numeric type}
	\begin{tabular}{lrrrrrr}
		\hline
		 &  & Mean Difference & SE & t & Cohen's d & p$_{bonf}$  \\
		\hline
		CHINESE & DOT & -0.073 & 0.032 & -2.259 & -0.505 & 0.153  \\
		  & ARABIC & -0.042 & 0.024 & -1.735 & -0.270 & 0.510  \\
		 & THAI & 0.059 & 0.026 & 2.271 & 0.393 & 0.148  \\
		DOT & ARIBIC & 0.030 & 0.032 & 0.951 & 0.221 & 1.000  \\
		  & THAI & 0.132 & 0.033 & 3.988 & 1.046 & $<$ .001  \\
		ARABIC & THAI & 0.101 & 0.026 & 3.969 & 0.702 & $<$ .001  \\
		\hline
	\end{tabular} 
	\label{tab:LangMatchedAcc_Cross}
\end{table}


\begin{table}[ht]
	\centering
	\caption{Post Hoc Comparisons - Numeric type}
	\begin{tabular}{lrrrrrr}
		\hline
		 &  & Mean Difference & SE & t & Cohen's d & p$_{bonf}$  \\
		\hline
		ARABIC & DOTS & -0.098 & 0.036 & -2.746 & -0.586 & 0.073  \\
		  & CHINESE & -0.059 & 0.042 & -1.424 & -0.304 & 1.000  \\
		 & THAI & -0.091 & 0.045 & -2.026 & -0.432 & 0.334  \\
		DOTS & CHINESE & 0.038 & 0.026 & 1.469 & 0.313 & 0.940  \\
		  & THAI & 0.007 & 0.038 & 0.183 & 0.039 & 1.000  \\
		CHINESE & THAI & -0.031 & 0.040 & -0.794 & -0.169 & 1.000  \\
		\hline
	\end{tabular} 
\label{app:ttest_comparison}
\end{table}



\section{Response accuracy and response bias} 
\label{Appendix:Bias}
%\lhead{Supplementary Material \ref{Appendix:E_CrossWheelTask}. \emph{Response bias}} 

Figure \ref{fig:AccRspFq_Cross}.a. shows the positive relationship (rank-order correlation $r$ = .61, $p$ $<$ .05) between response-frequency (blue) and response-accuracy (orange) for Arabic numerals in participant S1. This figure shows that, as the frequency of responding with a specific numeral increases, so too does identification accuracy  (\eg Arabic numeral 7). Similarly, as response frequency decreases, response accuracy also decreases (\eg Arabic numeral 5). This plot clearly shows the relationship between response-frequency (strength in Luce's model) and identification accuracy.

The dotted blue line in Figure \ref{fig:AccRspFq_Cross}.a represents a response-frequency matching the number of stimulus presentations. For example, a `6' response was made nearly as often as `6' was presented, however, these responses were correct only half of the time. By contrast, a `7' response was made nearly twice as often as it was presented with 50\% accuracy, showing an effect of response-bias. It is unclear what effect this response-bias had on the identification accuracy of each stimulus. This ambiguity necessitates the use of Luce's choice model. The relationship between response-frequency and response-accuracy is clearest on a participant-by-stimulus basis and has been represented for each participant and numeric-type as a scatter plot in Figure \ref{fig:AccRspFq_Cross}.b.

This figure highlights the effect of response-bias on response-accuracy. As an example, the number `7' has above average conditional accuracy, however, this participant responds with the number `7' more often than any other number. Does the conditional accuracy for number `7' reflect numeric identifiability, or, merely a bias in responding? The relationship between response-frequency and response-accuracy is clearest on a participant-by-stimulus basis, and has been represented for each language-type as a scatter plot in Figure \ref{fig:AccRspFq_Cross}.b.


\begin{figure}[tbh]
\centering \includegraphics[width = \linewidth]{Figures/Appendix/AppE/AccuracyResponseFreq.jpg}
\caption{a) Response frequency by response accuracy for participant S1, Arabic numerals. b) Scatter plot depicting a positive correlation between mean stimulus accuracy and response frequency, across language-types. c) Response frequency by response accuracy for stimuli for the Arabic (left), Chinese (mid-left), Thai (mid-right) and Dot (right) language-types. * $p$ $<$ .05, ** $p$ $<$ .01, *** $p$ $<$ .001}
\label{fig:AccRspFq_Cross}
\end{figure}

Figure \ref{fig:AccRspFq_Cross}.b. depicts the response-frequency by response-accuracy for each stimulus in each language-type. There is a strong positive correlation ($r$ = .56, $p$ $<$ .001) between response-frequency and response accuracy across all language-types. Figure \ref{fig:AccRspFq_Cross}.c shows the mean response-frequency and mean response-accuracy of each stimulus for each language-type. Averaging response-frequency and accuracy diminishes their correlation, however, clearly illustrates response patterns and accuracy for each stimulus. Together, these results show the positive relationship between accuracy and response-frequency, and motivate the need to remove response-bias before the assessment of the underlying mental space. 

\section{Scree analysis of bias-free MDS stress values} 
\label{Appendix:MDS}
%\lhead{Supplementary \ref{Appendix:E_CrossWheelTask}. \emph{MDS}} 

Scree analysis compares the multidimensional stress values (y-axis) against the number of MDS dimensions (x-axis). Scree analysis, such as this, is a subjective measure. A useful heuristic for identifying the correct number of dimensions is to look for the `elbow' where an increase in dimensionality does not meaningfully improve stress values. This elbow has been identified by a marker in each plot.

\subsection{Scree Plots} 
\begin{figure}[tbh]
\centering \includegraphics[scale = .6]{Figures/Appendix/AppE/MDSunbiasScree_1.jpg}
\caption{Bias-free MDS scree plots for Arabic digits (blue) and symbolic dots (green). The y-axis displays stress values, and the x-axis the number of dimensions. Markers identify the optimal number of dimensions in each scree plot.}
\label{fig:Apx_ScreeEngDot_Cross}
\end{figure}

\begin{figure}[tbh]
\centering \includegraphics[scale = .6]{Figures/Appendix/AppE/MDSunbiasScree_2.jpg}
\caption{Bias-free MDS scree plots for Chinese (red) and Thai (purple) symbols. The y-axis displays stress values, and the x-axis the number of dimensions. Markers identify the optimal number of dimensions in each scree plot.}
\label{fig:Apx_ScreeChnThi_Cross}
\end{figure}

\clearpage
\subsection{Individual MDS solutions} 

\begin{figure}[tbh]
\centering \includegraphics[scale = .67]{Figures/Appendix/AppE/BiasFree_Indiv_MDS_1.jpg}
\caption{Individual bias-free MDS solutions for the Arabic digits.}
\label{fig:Apx_MDSenglish_Cross}
\end{figure}

\begin{figure}[tbh]
\centering \includegraphics[scale = .67]{Figures/Appendix/AppE/BiasFree_Indiv_MDS_2.jpg}
\caption{Individual bias-free MDS solutions for symbolic dots. Dots are represented by Arabic numbers for simplicity.}
\label{fig:Apx_MDSdots_Cross}
\end{figure}

\begin{figure}[tbh]
\centering \includegraphics[scale = .67]{Figures/Appendix/AppE/BiasFree_Indiv_MDS_3.jpg}
\caption{Individual bias-free MDS solutions for Chinese symbols.}
\label{fig:Apx_MDSchinese_Cross}
\end{figure}

\begin{figure}[tbh]
\centering \includegraphics[scale = .67]{Figures/Appendix/AppE/BiasFree_Indiv_MDS_4.jpg}
\caption{Individual bias-free MDS solutions for the Thai symbols.}
\label{fig:Apx_MDSthai_Cross}
\end{figure}



\begin{figure}[tbh]
\centering \includegraphics[scale = .67]{Figures/Appendix/AppE/Biased_Indiv_MDS_1.jpg}
\caption{Individual biased MDS solutions for the Arabic digits.}
\label{fig:Apx_MDSenglishBiased_Cross}
\end{figure}

\begin{figure}[tbh]
\centering \includegraphics[scale = .67]{Figures/Appendix/AppE/Biased_Indiv_MDS_2.jpg}
\caption{Individual biased MDS solutions for symbolic dots. Dots are represented by Arabic numbers for simplicity.}
\label{fig:Apx_MDSdotsBiased_Cross}
\end{figure}

\begin{figure}[tbh]
\centering \includegraphics[scale = .67]{Figures/Appendix/AppE/Biased_Indiv_MDS_3.jpg}
\caption{Individual bias-free MDS solutions for Chinese symbols.}
\label{fig:Apx_MDSchineseBiased_Cross}
\end{figure}

\begin{figure}[tbh]
\centering \includegraphics[scale = .67]{Figures/Appendix/AppE/Biased_Indiv_MDS_4.jpg}
\caption{Individual bias-free MDS solutions for Thai symbols.}
\label{fig:Apx_MDSthaiBiased_Cross}
\end{figure}


\clearpage
\section{MDS cluster frequency heatmaps}
\label{Appendix:ClusterFq}
%\lhead{Supplementary \ref{Appendix:E_CrossWheelTask}. \emph{Cluster frequency heatmaps}} 


\begin{figure}[tbh]
\centering \includegraphics[scale = .7]{Figures/Appendix/AppE/Participant_2D_KM.jpg}
\caption{Proportional cluster-frequency heatmap for participants with two-dimensional MDS solutions, across 2--6 K-mean clusters. Larger proportions (darker colored squares) indicate items which most frequently cluster together. As opposed to the indscal heatmap in the main text, here, cluster frequencies are calculated separately for each participant, and an overall proportion presented.}
\label{fig:Apx_2Dheatmap_Cross}
\end{figure}

\clearpage
\section{Ideal observer comparison}
\label{Appendix:IOA}
%\lhead{Supplementary \ref{Appendix:E_CrossWheelTask}. \emph{Ideal observer}} 

The ideal observer analysis is a simple template matching process that compares numeric stimuli, pixel-by-pixel, to generate a confusion matrix. The ideal observer is not a model of human performance, but rather, a benchmark against which we may compare the performance of human observers \cite<\eg>{gold1999identification, eidels2014measuring}. The `ideal observer' compares a noisy numeric stimulus to all possible templates, for example, comparing a noisy `1' stimulus to the numerals `1--9'. The template with the best cross-correlational match over many iterations, with randomly sampled noise is selected as the `ideal observer response'. Normally distributed noise ($\mu$ = 0, $\sigma$ = [1.065, .12, 1.127, 1.463] for Arabic, dot, Chinese and Thai numerals, respectively) is added to each numeric stimulus, until the ideal observer's accuracy resembles the average accuracy of the participants. This process was repeated 10,000 times, per numeric-stimulus, per numeric-type, generating four confusion matrices. For direct comparison to the empirical data, Luce's choice model was then applied to these confusion scores.

% TO DO...AE
Figure \ref{fig:IOA2}.a displays the ideal observer MDS results for each numeric type, while Figure \ref{fig:IOA2}.b displays the corresponding K-mean cluster frequency heatmap. For Arabic numerals, the ideal observer and Chinese speaking cohort shared similar MDS and cluster patterns for the subset of numerals [2, 7], [5, 6], [5, 9] and [6, 9]. For non-symbolic dots, shared MDS and cluster patterns included numerals [1, 3] and [6, 7]. For Chinese numerals, shared MDS and cluster patterns included numerals [1, 6] and [2, 3]. Finally, for Thai numerals, shared MDS and cluster patterns included numerals [3, 7] and [4, 5].

\begin{figure}[tbh]
\centering \includegraphics[scale = .4]{Figures/Appendix/AppE/IOA2.jpg}
\caption{a) Ideal observer analysis bias-free MDS solutions, generated separately for each numeric-type. Non-symbolic dots are displayed as Arabic numerals in the MDS plot for clarity to the reader. b) Ideal observer K-mean cluster frequency heatmaps.}
\label{fig:IOA2}
\end{figure}

The ideal observer does not share many commonalities with the empirical data of the Chinese speaking cohort. This observation may be caused by several factors. First, the ideal observer is a template matching process that only handles extant features. Arabic and Thai numerals appeared to be confused along a dimension of openness. Openness is the concave absence within a shape and it is not an extant feature, therefore it is poorly captured by our template matching process.

The Chinese speaking cohort appeared to confused non-symbolic dots along dimensions of perceptual similarity and numerical proximity. The ideal observer holds no concept of `numerical proximity', merely perceptual similarity. Lacking this extra dimensionality, it should be expected that the ideal observer results would differ from our empirical findings. 

Finally, the Chinese speaking cohort appeared to confused Chinese numerals based upon individual line-strokes, specifically the number of horizontal strokes and the stroke curvature. These are highly specific features and would form only part of the template matching process carried out by the ideal observer. Expertise with the character set allows Chinese speakers to focus upon highly distinguishable character features --- something not afforded to our simple observer. It is for this reason we speculate that a difference exists between the ideal observer and Chinese speaking cohort for Chinese numerals.

The ideal observer, as described here, is limited in a number of ways. Although a `better' observer could be developed, for example, if one employed a machine learning algorithm; developing a `human-like' observer was not the purpose of this exercise. Instead, this procedure was simply to provide a benchmark comparison with which to compare performance given items were confused only due to perceptual similarities. To this end, the ideal observer completed its task.

\pagebreak