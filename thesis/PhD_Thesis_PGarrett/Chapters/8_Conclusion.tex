\chapter{General conclusions} 

\label{Chapter 8}

\lhead{Chapter 8. \emph{General conclusions}}

This thesis offers a number of contributions to the literature on numerical cognition. In the first research stream, the comparison of quantities (Chapters 2--5), I examined the cognitive processes with which we estimate a single item-set (Chapter 2) and applied these findings to study the processing architecture and workload capacity of a comparative estimation system (Chapter 3). This work led to a subsequent investigation examining the systems of comparative subitizing (Chapter 4). In these studies, I applied Systems Factorial Technology to analyse response times and uncover basic properties of the underlying processing system. I observed empirical SIC signatures that indicated a mixture of two different processing architectures. To identify these mixtures, I completed a simulation study that extended the canonical SIC signatures to systematic mixtures of processing architecture and workload capacity (Chapter 5). In the second research stream, the confusion of numbers (Chapters 6 \& 7), I examined the mental representations of familiar and unfamiliar numerals within an English (Chapter 6) and Chinese (Chapter 7) speaking cohorts. Across both cohorts, I found that symbolic numerals were represented along dimensions of perceptual similarity, and non-symbolic numerals along dimensions of perceptual similarity and numerical proximity. Finally, I found that expertise may change the way we perceive numerals and thereby shape our mental representations. 
%AE: fine, but you know i think these conclusions are not grounded
%PG: I know, but I'm going with it :p We'll just need to see what the reviewers think

Each of the above thesis chapters have worked in concert to extend our understanding of numerical cognition. Together, these chapters formed two complimentary research streams: one that investigated the enumeration of non-symbolic quantities, and one that investigated the representation of symbolic numerals. Although I have discussed these streams as independent components, a complete understanding of numerical cognition requires the simultaneous investigation of both symbolic and non-symbolic quantities.
% AE: I edited last para' so check that you agree
% PG: Great, thanks Ami

Numerical cognition encompasses the communication of quantity through coarse non-symbolic values, for example, tallies and dots, and the precise communication of symbolic digits. Numerical cognition describes our mental representations of number and quantity, and how these constructs are informed by our past experiences and current levels of expertise. Finally, numerical cognition is the complimentary understanding of both quantities \textit{and} numbers, as epitomized by my complimentary research streams. Together, these two research streams aimed to extend our scientific understanding of numerical cognition through the examination of both quantities and numbers.

%AE: do you mean complementary ? 
% PG: Yes, thanks for that!
The research chapters and complementary research streams of this thesis have made several substantial contributions to the literature. The research chapters of this thesis have extended on previous methodological designs (Chapters 2, 3, 4, and 6), developed new analysis techniques (Chapters 5 and 6), and provided new insights on the fundamental cognitive processes that we share across cultures (Chapter 7). In each of these chapters, I have discussed these contributions in detail. Rather than retread these points, I will now discuss the general implications and the future directions that stem from this thesis. 

%AE: i substituted research component with 'research strand'; for consistency perhaps do this in ALL of the thesis?
%PG: I've gone with stream - same use, I just like it more than strand.

\section{Implications and future directions}
\subsection{Systems of enumeration}
The work reported in the current thesis has many implications for the field of numerical cognition. In the first research stream, I considered systems of estimation and subitizing. I observed that estimation systems operate under predominantly parallel processing architectures, and that subitizing systems operate under predominantly serial processing architectures. In both instances, these systems were less-efficient than the predictions of a theoretical serial processing system. I suggested that this was due to context effects. This work is important, not only for the design of future research, but for the interpretation of past findings. 

Studies that investigate processes of enumeration often display an array containing two or more item sub-sets \cite<e.g., green, red and blue colored discs;>{HALBERDA_2006}. These studies generally work under the implied assumption of context invariance --- that the cognitive process operating on one item-set is unaffected by the presence of another. Our work clearly violates this assumption for both small and large quantities. This means that past and future research that relies on the interpretation of response-times, must consider the inhibitory influence of these context effects within the domain of enumeration. Future research can examine what these context effects are, and how they can be ameliorated.

\subsection{Systems factorial technology}
Systems factorial technology and the associated measure the SIC($t$), has been used by Townsend and colleagues \cite{Townsend_1995, Townsend_2004, eidels2010stroop, Fific2010, houpt2010statSIC, Little2011, Little2017} to diagnose processing architecture in a range of experimental tasks. Typically, the SIC is compared to canonical signatures and is used to diagnose processing architecture (parallel vs serial) and stopping-rule (minimum-time vs maximum-time). In Chapter 5, I developed a class of mixture models and reported new SIC signatures, corresponding to systematic mixtures of processing architecture. 

Our new mixture SIC($t$) signatures have implications for both past and future work. Previously, SIC($t$) signatures that did not match  a single canonical model were either i) not diagnosed, ii) diagnosed by the best matching canonical model, or more recently iii) fit parametrically as a mixture or contaminant model \cite{Moneer2016}. By providing a new set of signatures for mixture models, SIC(\textit{t}) functions can be diagnosed by their component architectures and relative proportions. Future research will need to consider non-parametric statistical methods through which these mixture models may be diagnosed. This would add to the growing suite of statistical tools \cite<\eg>{houpt2010statSIC, houpt2012statCap, houpt2017bayesSIC, houpt2017hierarchical, Moneer2016} used by the SFT community for diagnosing system properties. This work might also extend to diagnosing mixtures of stopping-rule \cite{TillmanStopping}. 

\subsection{Numeral identification}
In the second research stream, the confusion of numbers, I investigated the mental representation of numerals across English and Chinese speakers. Across these two cohorts, I found that mental representations changed in response to different levels of numeric expertise. Importantly, when numeric expertise was the same between cohorts, mental representations were nearly identical. This work presents new questions and potential lines of inquiry regarding the effect of expertise on our mental representations. 

Future research could examine how experience alters the proximity of items within the mental space. For example, does the frequency of exposure to unfamiliar numerals shift items in the mental space? Or, does a shift in the mental space require unfamiliar numerals to be processed at a deeper cognitive level, for example, by using them for arithmetic operations? Furthermore, what are the temporal constraints to these processes. For example, how long do changes to the mental space take, and do these changes affect different dimensions at different rates? These questions could be applied to any investigation of expertise using confusion patterns, however, as we have shown, numerals appear to be particularly well suited to this task. 

In a second line of future research, I propose a final extension of the work in Chapter 6 and 7 by examining the mental representation of numerals within a Thai cohort. In both of the tested cohorts, Thai numerals served as a symbolic control, with neither cohort having any experience with the numeric set. Replicating our investigation with a Thai cohort would give deeper insight into i) whether Thai numerals shift in the mental space with experience, and ii) further validate that non-symbolic numerals are represented identically across cultures. If this cohort had little expertise with Arabic numerals, this investigation might address iii) whether similarities between Australian and Taiwanese representations for Arabic numerals reflected numeric expertise. 

Finally, my work on confusion analysis not only represents an important contribution to numerical cognition, but a contribution to methodological design. In this work, I developed a rapid staircase procedure, applied a method for removing response-bias, modelled group level MDS data and examined MDS distances through an objective clustering technique. The combination of these experimental and analytic methods produced clear, illustrative results that were easy to compare within and between our cohorts. Importantly, these principled methods can be applied in any investigation of confusion data. 


\section{Conclusion}
The ability to compare quantities and symbolic numerals has shaped the course of human history. Every aspect of human culture, from trade and science, to mathematics and war, has been advanced by our understanding of quantities and numbers. In the modern era, we regularly compare two quantities, (\eg number of people in coffee queues), and communicate symbolic value, (\eg phone numbers and money). Until recently, the cognitive processes that underpin such abilities have been poorly understood. The aim of this thesis was to provide new insights into the fundamental cognitive processes that govern our comparisons of quantity and our confusion of numbers. Over the course of the past seven chapters, I addressed this aim by developing theoretically driven experimental designs, analyses and simulations.In accordance with the aim of this thesis, I have shed new light on the nature of the cognitive processes that are fundamental to our understanding of quantity and number. Yet, so much more remains to be learned. Future researchers have much to explore on the nature of numerical cognition. I am sure the findings of these studies will be both enlightening and numerous.

%AMI DONE 1/11/2019